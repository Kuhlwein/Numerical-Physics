\documentclass[a4paper,twocolumn]{article}
\usepackage{amsmath,amssymb}
\usepackage{kpfonts}

\title{The Bessel function of the first kind}
\date{}
\author{Morten Kuhlwein}
\setlength\parindent{0pt}

\begin{document}
	\maketitle
	\section*{Exercise 25}
	In this exercise the Bessel function of the first kind of integer index is implemented.
	For this purpose the integral representation of the Besselfunction is used:
	\begin{equation}
		J_{n} (x) = \frac{1}{\pi} \int_{0}^{\pi} \cos{(n t - x \sin{(t)})} \mathop{\mathrm{d} t}.
	\end{equation}
	
	Given an integer index $ n $, and a value $ x $, the value of $ J_{n} (x) $ can be found by solving the integral.
	For this purpose the numerical integration routines from the Gnu Scientific Library is used.

	The Bessel functions for the integer indexes $ n = 1, 2 $ and $ 3 $ found using this method can be found on figure \ref{fig:plot}, along with values calculated using the built-in Bessel functions from {\tt<math.h>}.


	\begin{figure*}[t]
		\centering
		% GNUPLOT: LaTeX picture with Postscript
\begingroup%
\makeatletter%
\newcommand{\GNUPLOTspecial}{%
  \@sanitize\catcode`\%=14\relax\special}%
\setlength{\unitlength}{0.0500bp}%
\begin{picture}(8640,6048)(0,0)%
  {\GNUPLOTspecial{"
%!PS-Adobe-2.0 EPSF-2.0
%%Title: plot.tex
%%Creator: gnuplot 5.2 patchlevel 2
%%CreationDate: Wed Mar 21 17:07:01 2018
%%DocumentFonts: 
%%BoundingBox: 0 0 432 302
%%EndComments
%%BeginProlog
/gnudict 256 dict def
gnudict begin
%
% The following true/false flags may be edited by hand if desired.
% The unit line width and grayscale image gamma correction may also be changed.
%
/Color false def
/Blacktext true def
/Solid false def
/Dashlength 1 def
/Landscape false def
/Level1 false def
/Level3 false def
/Rounded false def
/ClipToBoundingBox false def
/SuppressPDFMark false def
/TransparentPatterns false def
/gnulinewidth 5.000 def
/userlinewidth gnulinewidth def
/Gamma 1.0 def
/BackgroundColor {-1.000 -1.000 -1.000} def
%
/vshift -66 def
/dl1 {
  10.0 Dashlength userlinewidth gnulinewidth div mul mul mul
  Rounded { currentlinewidth 0.75 mul sub dup 0 le { pop 0.01 } if } if
} def
/dl2 {
  10.0 Dashlength userlinewidth gnulinewidth div mul mul mul
  Rounded { currentlinewidth 0.75 mul add } if
} def
/hpt_ 31.5 def
/vpt_ 31.5 def
/hpt hpt_ def
/vpt vpt_ def
/doclip {
  ClipToBoundingBox {
    newpath 0 0 moveto 432 0 lineto 432 302 lineto 0 302 lineto closepath
    clip
  } if
} def
%
% Gnuplot Prolog Version 5.1 (Oct 2015)
%
%/SuppressPDFMark true def
%
/M {moveto} bind def
/L {lineto} bind def
/R {rmoveto} bind def
/V {rlineto} bind def
/N {newpath moveto} bind def
/Z {closepath} bind def
/C {setrgbcolor} bind def
/f {rlineto fill} bind def
/g {setgray} bind def
/Gshow {show} def   % May be redefined later in the file to support UTF-8
/vpt2 vpt 2 mul def
/hpt2 hpt 2 mul def
/Lshow {currentpoint stroke M 0 vshift R 
	Blacktext {gsave 0 setgray textshow grestore} {textshow} ifelse} def
/Rshow {currentpoint stroke M dup stringwidth pop neg vshift R
	Blacktext {gsave 0 setgray textshow grestore} {textshow} ifelse} def
/Cshow {currentpoint stroke M dup stringwidth pop -2 div vshift R 
	Blacktext {gsave 0 setgray textshow grestore} {textshow} ifelse} def
/UP {dup vpt_ mul /vpt exch def hpt_ mul /hpt exch def
  /hpt2 hpt 2 mul def /vpt2 vpt 2 mul def} def
/DL {Color {setrgbcolor Solid {pop []} if 0 setdash}
 {pop pop pop 0 setgray Solid {pop []} if 0 setdash} ifelse} def
/BL {stroke userlinewidth 2 mul setlinewidth
	Rounded {1 setlinejoin 1 setlinecap} if} def
/AL {stroke userlinewidth 2 div setlinewidth
	Rounded {1 setlinejoin 1 setlinecap} if} def
/UL {dup gnulinewidth mul /userlinewidth exch def
	dup 1 lt {pop 1} if 10 mul /udl exch def} def
/PL {stroke userlinewidth setlinewidth
	Rounded {1 setlinejoin 1 setlinecap} if} def
3.8 setmiterlimit
% Classic Line colors (version 5.0)
/LCw {1 1 1} def
/LCb {0 0 0} def
/LCa {0 0 0} def
/LC0 {1 0 0} def
/LC1 {0 1 0} def
/LC2 {0 0 1} def
/LC3 {1 0 1} def
/LC4 {0 1 1} def
/LC5 {1 1 0} def
/LC6 {0 0 0} def
/LC7 {1 0.3 0} def
/LC8 {0.5 0.5 0.5} def
% Default dash patterns (version 5.0)
/LTB {BL [] LCb DL} def
/LTw {PL [] 1 setgray} def
/LTb {PL [] LCb DL} def
/LTa {AL [1 udl mul 2 udl mul] 0 setdash LCa setrgbcolor} def
/LT0 {PL [] LC0 DL} def
/LT1 {PL [2 dl1 3 dl2] LC1 DL} def
/LT2 {PL [1 dl1 1.5 dl2] LC2 DL} def
/LT3 {PL [6 dl1 2 dl2 1 dl1 2 dl2] LC3 DL} def
/LT4 {PL [1 dl1 2 dl2 6 dl1 2 dl2 1 dl1 2 dl2] LC4 DL} def
/LT5 {PL [4 dl1 2 dl2] LC5 DL} def
/LT6 {PL [1.5 dl1 1.5 dl2 1.5 dl1 1.5 dl2 1.5 dl1 6 dl2] LC6 DL} def
/LT7 {PL [3 dl1 3 dl2 1 dl1 3 dl2] LC7 DL} def
/LT8 {PL [2 dl1 2 dl2 2 dl1 6 dl2] LC8 DL} def
/SL {[] 0 setdash} def
/Pnt {stroke [] 0 setdash gsave 1 setlinecap M 0 0 V stroke grestore} def
/Dia {stroke [] 0 setdash 2 copy vpt add M
  hpt neg vpt neg V hpt vpt neg V
  hpt vpt V hpt neg vpt V closepath stroke
  Pnt} def
/Pls {stroke [] 0 setdash vpt sub M 0 vpt2 V
  currentpoint stroke M
  hpt neg vpt neg R hpt2 0 V stroke
 } def
/Box {stroke [] 0 setdash 2 copy exch hpt sub exch vpt add M
  0 vpt2 neg V hpt2 0 V 0 vpt2 V
  hpt2 neg 0 V closepath stroke
  Pnt} def
/Crs {stroke [] 0 setdash exch hpt sub exch vpt add M
  hpt2 vpt2 neg V currentpoint stroke M
  hpt2 neg 0 R hpt2 vpt2 V stroke} def
/TriU {stroke [] 0 setdash 2 copy vpt 1.12 mul add M
  hpt neg vpt -1.62 mul V
  hpt 2 mul 0 V
  hpt neg vpt 1.62 mul V closepath stroke
  Pnt} def
/Star {2 copy Pls Crs} def
/BoxF {stroke [] 0 setdash exch hpt sub exch vpt add M
  0 vpt2 neg V hpt2 0 V 0 vpt2 V
  hpt2 neg 0 V closepath fill} def
/TriUF {stroke [] 0 setdash vpt 1.12 mul add M
  hpt neg vpt -1.62 mul V
  hpt 2 mul 0 V
  hpt neg vpt 1.62 mul V closepath fill} def
/TriD {stroke [] 0 setdash 2 copy vpt 1.12 mul sub M
  hpt neg vpt 1.62 mul V
  hpt 2 mul 0 V
  hpt neg vpt -1.62 mul V closepath stroke
  Pnt} def
/TriDF {stroke [] 0 setdash vpt 1.12 mul sub M
  hpt neg vpt 1.62 mul V
  hpt 2 mul 0 V
  hpt neg vpt -1.62 mul V closepath fill} def
/DiaF {stroke [] 0 setdash vpt add M
  hpt neg vpt neg V hpt vpt neg V
  hpt vpt V hpt neg vpt V closepath fill} def
/Pent {stroke [] 0 setdash 2 copy gsave
  translate 0 hpt M 4 {72 rotate 0 hpt L} repeat
  closepath stroke grestore Pnt} def
/PentF {stroke [] 0 setdash gsave
  translate 0 hpt M 4 {72 rotate 0 hpt L} repeat
  closepath fill grestore} def
/Circle {stroke [] 0 setdash 2 copy
  hpt 0 360 arc stroke Pnt} def
/CircleF {stroke [] 0 setdash hpt 0 360 arc fill} def
/C0 {BL [] 0 setdash 2 copy moveto vpt 90 450 arc} bind def
/C1 {BL [] 0 setdash 2 copy moveto
	2 copy vpt 0 90 arc closepath fill
	vpt 0 360 arc closepath} bind def
/C2 {BL [] 0 setdash 2 copy moveto
	2 copy vpt 90 180 arc closepath fill
	vpt 0 360 arc closepath} bind def
/C3 {BL [] 0 setdash 2 copy moveto
	2 copy vpt 0 180 arc closepath fill
	vpt 0 360 arc closepath} bind def
/C4 {BL [] 0 setdash 2 copy moveto
	2 copy vpt 180 270 arc closepath fill
	vpt 0 360 arc closepath} bind def
/C5 {BL [] 0 setdash 2 copy moveto
	2 copy vpt 0 90 arc
	2 copy moveto
	2 copy vpt 180 270 arc closepath fill
	vpt 0 360 arc} bind def
/C6 {BL [] 0 setdash 2 copy moveto
	2 copy vpt 90 270 arc closepath fill
	vpt 0 360 arc closepath} bind def
/C7 {BL [] 0 setdash 2 copy moveto
	2 copy vpt 0 270 arc closepath fill
	vpt 0 360 arc closepath} bind def
/C8 {BL [] 0 setdash 2 copy moveto
	2 copy vpt 270 360 arc closepath fill
	vpt 0 360 arc closepath} bind def
/C9 {BL [] 0 setdash 2 copy moveto
	2 copy vpt 270 450 arc closepath fill
	vpt 0 360 arc closepath} bind def
/C10 {BL [] 0 setdash 2 copy 2 copy moveto vpt 270 360 arc closepath fill
	2 copy moveto
	2 copy vpt 90 180 arc closepath fill
	vpt 0 360 arc closepath} bind def
/C11 {BL [] 0 setdash 2 copy moveto
	2 copy vpt 0 180 arc closepath fill
	2 copy moveto
	2 copy vpt 270 360 arc closepath fill
	vpt 0 360 arc closepath} bind def
/C12 {BL [] 0 setdash 2 copy moveto
	2 copy vpt 180 360 arc closepath fill
	vpt 0 360 arc closepath} bind def
/C13 {BL [] 0 setdash 2 copy moveto
	2 copy vpt 0 90 arc closepath fill
	2 copy moveto
	2 copy vpt 180 360 arc closepath fill
	vpt 0 360 arc closepath} bind def
/C14 {BL [] 0 setdash 2 copy moveto
	2 copy vpt 90 360 arc closepath fill
	vpt 0 360 arc} bind def
/C15 {BL [] 0 setdash 2 copy vpt 0 360 arc closepath fill
	vpt 0 360 arc closepath} bind def
/Rec {newpath 4 2 roll moveto 1 index 0 rlineto 0 exch rlineto
	neg 0 rlineto closepath} bind def
/Square {dup Rec} bind def
/Bsquare {vpt sub exch vpt sub exch vpt2 Square} bind def
/S0 {BL [] 0 setdash 2 copy moveto 0 vpt rlineto BL Bsquare} bind def
/S1 {BL [] 0 setdash 2 copy vpt Square fill Bsquare} bind def
/S2 {BL [] 0 setdash 2 copy exch vpt sub exch vpt Square fill Bsquare} bind def
/S3 {BL [] 0 setdash 2 copy exch vpt sub exch vpt2 vpt Rec fill Bsquare} bind def
/S4 {BL [] 0 setdash 2 copy exch vpt sub exch vpt sub vpt Square fill Bsquare} bind def
/S5 {BL [] 0 setdash 2 copy 2 copy vpt Square fill
	exch vpt sub exch vpt sub vpt Square fill Bsquare} bind def
/S6 {BL [] 0 setdash 2 copy exch vpt sub exch vpt sub vpt vpt2 Rec fill Bsquare} bind def
/S7 {BL [] 0 setdash 2 copy exch vpt sub exch vpt sub vpt vpt2 Rec fill
	2 copy vpt Square fill Bsquare} bind def
/S8 {BL [] 0 setdash 2 copy vpt sub vpt Square fill Bsquare} bind def
/S9 {BL [] 0 setdash 2 copy vpt sub vpt vpt2 Rec fill Bsquare} bind def
/S10 {BL [] 0 setdash 2 copy vpt sub vpt Square fill 2 copy exch vpt sub exch vpt Square fill
	Bsquare} bind def
/S11 {BL [] 0 setdash 2 copy vpt sub vpt Square fill 2 copy exch vpt sub exch vpt2 vpt Rec fill
	Bsquare} bind def
/S12 {BL [] 0 setdash 2 copy exch vpt sub exch vpt sub vpt2 vpt Rec fill Bsquare} bind def
/S13 {BL [] 0 setdash 2 copy exch vpt sub exch vpt sub vpt2 vpt Rec fill
	2 copy vpt Square fill Bsquare} bind def
/S14 {BL [] 0 setdash 2 copy exch vpt sub exch vpt sub vpt2 vpt Rec fill
	2 copy exch vpt sub exch vpt Square fill Bsquare} bind def
/S15 {BL [] 0 setdash 2 copy Bsquare fill Bsquare} bind def
/D0 {gsave translate 45 rotate 0 0 S0 stroke grestore} bind def
/D1 {gsave translate 45 rotate 0 0 S1 stroke grestore} bind def
/D2 {gsave translate 45 rotate 0 0 S2 stroke grestore} bind def
/D3 {gsave translate 45 rotate 0 0 S3 stroke grestore} bind def
/D4 {gsave translate 45 rotate 0 0 S4 stroke grestore} bind def
/D5 {gsave translate 45 rotate 0 0 S5 stroke grestore} bind def
/D6 {gsave translate 45 rotate 0 0 S6 stroke grestore} bind def
/D7 {gsave translate 45 rotate 0 0 S7 stroke grestore} bind def
/D8 {gsave translate 45 rotate 0 0 S8 stroke grestore} bind def
/D9 {gsave translate 45 rotate 0 0 S9 stroke grestore} bind def
/D10 {gsave translate 45 rotate 0 0 S10 stroke grestore} bind def
/D11 {gsave translate 45 rotate 0 0 S11 stroke grestore} bind def
/D12 {gsave translate 45 rotate 0 0 S12 stroke grestore} bind def
/D13 {gsave translate 45 rotate 0 0 S13 stroke grestore} bind def
/D14 {gsave translate 45 rotate 0 0 S14 stroke grestore} bind def
/D15 {gsave translate 45 rotate 0 0 S15 stroke grestore} bind def
/DiaE {stroke [] 0 setdash vpt add M
  hpt neg vpt neg V hpt vpt neg V
  hpt vpt V hpt neg vpt V closepath stroke} def
/BoxE {stroke [] 0 setdash exch hpt sub exch vpt add M
  0 vpt2 neg V hpt2 0 V 0 vpt2 V
  hpt2 neg 0 V closepath stroke} def
/TriUE {stroke [] 0 setdash vpt 1.12 mul add M
  hpt neg vpt -1.62 mul V
  hpt 2 mul 0 V
  hpt neg vpt 1.62 mul V closepath stroke} def
/TriDE {stroke [] 0 setdash vpt 1.12 mul sub M
  hpt neg vpt 1.62 mul V
  hpt 2 mul 0 V
  hpt neg vpt -1.62 mul V closepath stroke} def
/PentE {stroke [] 0 setdash gsave
  translate 0 hpt M 4 {72 rotate 0 hpt L} repeat
  closepath stroke grestore} def
/CircE {stroke [] 0 setdash 
  hpt 0 360 arc stroke} def
/Opaque {gsave closepath 1 setgray fill grestore 0 setgray closepath} def
/DiaW {stroke [] 0 setdash vpt add M
  hpt neg vpt neg V hpt vpt neg V
  hpt vpt V hpt neg vpt V Opaque stroke} def
/BoxW {stroke [] 0 setdash exch hpt sub exch vpt add M
  0 vpt2 neg V hpt2 0 V 0 vpt2 V
  hpt2 neg 0 V Opaque stroke} def
/TriUW {stroke [] 0 setdash vpt 1.12 mul add M
  hpt neg vpt -1.62 mul V
  hpt 2 mul 0 V
  hpt neg vpt 1.62 mul V Opaque stroke} def
/TriDW {stroke [] 0 setdash vpt 1.12 mul sub M
  hpt neg vpt 1.62 mul V
  hpt 2 mul 0 V
  hpt neg vpt -1.62 mul V Opaque stroke} def
/PentW {stroke [] 0 setdash gsave
  translate 0 hpt M 4 {72 rotate 0 hpt L} repeat
  Opaque stroke grestore} def
/CircW {stroke [] 0 setdash 
  hpt 0 360 arc Opaque stroke} def
/BoxFill {gsave Rec 1 setgray fill grestore} def
/Density {
  /Fillden exch def
  currentrgbcolor
  /ColB exch def /ColG exch def /ColR exch def
  /ColR ColR Fillden mul Fillden sub 1 add def
  /ColG ColG Fillden mul Fillden sub 1 add def
  /ColB ColB Fillden mul Fillden sub 1 add def
  ColR ColG ColB setrgbcolor} def
/BoxColFill {gsave Rec PolyFill} def
/PolyFill {gsave Density fill grestore grestore} def
/h {rlineto rlineto rlineto gsave closepath fill grestore} bind def
%
% PostScript Level 1 Pattern Fill routine for rectangles
% Usage: x y w h s a XX PatternFill
%	x,y = lower left corner of box to be filled
%	w,h = width and height of box
%	  a = angle in degrees between lines and x-axis
%	 XX = 0/1 for no/yes cross-hatch
%
/PatternFill {gsave /PFa [ 9 2 roll ] def
  PFa 0 get PFa 2 get 2 div add PFa 1 get PFa 3 get 2 div add translate
  PFa 2 get -2 div PFa 3 get -2 div PFa 2 get PFa 3 get Rec
  TransparentPatterns {} {gsave 1 setgray fill grestore} ifelse
  clip
  currentlinewidth 0.5 mul setlinewidth
  /PFs PFa 2 get dup mul PFa 3 get dup mul add sqrt def
  0 0 M PFa 5 get rotate PFs -2 div dup translate
  0 1 PFs PFa 4 get div 1 add floor cvi
	{PFa 4 get mul 0 M 0 PFs V} for
  0 PFa 6 get ne {
	0 1 PFs PFa 4 get div 1 add floor cvi
	{PFa 4 get mul 0 2 1 roll M PFs 0 V} for
 } if
  stroke grestore} def
%
/languagelevel where
 {pop languagelevel} {1} ifelse
dup 2 lt
	{/InterpretLevel1 true def
	 /InterpretLevel3 false def}
	{/InterpretLevel1 Level1 def
	 2 gt
	    {/InterpretLevel3 Level3 def}
	    {/InterpretLevel3 false def}
	 ifelse }
 ifelse
%
% PostScript level 2 pattern fill definitions
%
/Level2PatternFill {
/Tile8x8 {/PaintType 2 /PatternType 1 /TilingType 1 /BBox [0 0 8 8] /XStep 8 /YStep 8}
	bind def
/KeepColor {currentrgbcolor [/Pattern /DeviceRGB] setcolorspace} bind def
<< Tile8x8
 /PaintProc {0.5 setlinewidth pop 0 0 M 8 8 L 0 8 M 8 0 L stroke} 
>> matrix makepattern
/Pat1 exch def
<< Tile8x8
 /PaintProc {0.5 setlinewidth pop 0 0 M 8 8 L 0 8 M 8 0 L stroke
	0 4 M 4 8 L 8 4 L 4 0 L 0 4 L stroke}
>> matrix makepattern
/Pat2 exch def
<< Tile8x8
 /PaintProc {0.5 setlinewidth pop 0 0 M 0 8 L
	8 8 L 8 0 L 0 0 L fill}
>> matrix makepattern
/Pat3 exch def
<< Tile8x8
 /PaintProc {0.5 setlinewidth pop -4 8 M 8 -4 L
	0 12 M 12 0 L stroke}
>> matrix makepattern
/Pat4 exch def
<< Tile8x8
 /PaintProc {0.5 setlinewidth pop -4 0 M 8 12 L
	0 -4 M 12 8 L stroke}
>> matrix makepattern
/Pat5 exch def
<< Tile8x8
 /PaintProc {0.5 setlinewidth pop -2 8 M 4 -4 L
	0 12 M 8 -4 L 4 12 M 10 0 L stroke}
>> matrix makepattern
/Pat6 exch def
<< Tile8x8
 /PaintProc {0.5 setlinewidth pop -2 0 M 4 12 L
	0 -4 M 8 12 L 4 -4 M 10 8 L stroke}
>> matrix makepattern
/Pat7 exch def
<< Tile8x8
 /PaintProc {0.5 setlinewidth pop 8 -2 M -4 4 L
	12 0 M -4 8 L 12 4 M 0 10 L stroke}
>> matrix makepattern
/Pat8 exch def
<< Tile8x8
 /PaintProc {0.5 setlinewidth pop 0 -2 M 12 4 L
	-4 0 M 12 8 L -4 4 M 8 10 L stroke}
>> matrix makepattern
/Pat9 exch def
/Pattern1 {PatternBgnd KeepColor Pat1 setpattern} bind def
/Pattern2 {PatternBgnd KeepColor Pat2 setpattern} bind def
/Pattern3 {PatternBgnd KeepColor Pat3 setpattern} bind def
/Pattern4 {PatternBgnd KeepColor Landscape {Pat5} {Pat4} ifelse setpattern} bind def
/Pattern5 {PatternBgnd KeepColor Landscape {Pat4} {Pat5} ifelse setpattern} bind def
/Pattern6 {PatternBgnd KeepColor Landscape {Pat9} {Pat6} ifelse setpattern} bind def
/Pattern7 {PatternBgnd KeepColor Landscape {Pat8} {Pat7} ifelse setpattern} bind def
} def
%
%
%End of PostScript Level 2 code
%
/PatternBgnd {
  TransparentPatterns {} {gsave 1 setgray fill grestore} ifelse
} def
%
% Substitute for Level 2 pattern fill codes with
% grayscale if Level 2 support is not selected.
%
/Level1PatternFill {
/Pattern1 {0.250 Density} bind def
/Pattern2 {0.500 Density} bind def
/Pattern3 {0.750 Density} bind def
/Pattern4 {0.125 Density} bind def
/Pattern5 {0.375 Density} bind def
/Pattern6 {0.625 Density} bind def
/Pattern7 {0.875 Density} bind def
} def
%
% Now test for support of Level 2 code
%
Level1 {Level1PatternFill} {Level2PatternFill} ifelse
%
/Symbol-Oblique /Symbol findfont [1 0 .167 1 0 0] makefont
dup length dict begin {1 index /FID eq {pop pop} {def} ifelse} forall
currentdict end definefont pop
%
Level1 SuppressPDFMark or 
{} {
/SDict 10 dict def
systemdict /pdfmark known not {
  userdict /pdfmark systemdict /cleartomark get put
} if
SDict begin [
  /Title (plot.tex)
  /Subject (gnuplot plot)
  /Creator (gnuplot 5.2 patchlevel 2)
%  /Producer (gnuplot)
%  /Keywords ()
  /CreationDate (Wed Mar 21 17:07:01 2018)
  /DOCINFO pdfmark
end
} ifelse
%
% Support for boxed text - Ethan A Merritt Sep 2016
%
/InitTextBox { userdict /TBy2 3 -1 roll put userdict /TBx2 3 -1 roll put
           userdict /TBy1 3 -1 roll put userdict /TBx1 3 -1 roll put
	   /Boxing true def } def
/ExtendTextBox { dup type /stringtype eq
    { Boxing { gsave dup false charpath pathbbox
      dup TBy2 gt {userdict /TBy2 3 -1 roll put} {pop} ifelse
      dup TBx2 gt {userdict /TBx2 3 -1 roll put} {pop} ifelse
      dup TBy1 lt {userdict /TBy1 3 -1 roll put} {pop} ifelse
      dup TBx1 lt {userdict /TBx1 3 -1 roll put} {pop} ifelse
      grestore } if }
    {} ifelse} def
/PopTextBox { newpath TBx1 TBxmargin sub TBy1 TBymargin sub M
               TBx1 TBxmargin sub TBy2 TBymargin add L
	       TBx2 TBxmargin add TBy2 TBymargin add L
	       TBx2 TBxmargin add TBy1 TBymargin sub L closepath } def
/DrawTextBox { PopTextBox stroke /Boxing false def} def
/FillTextBox { gsave PopTextBox fill grestore /Boxing false def} def
0 0 0 0 InitTextBox
/TBxmargin 20 def
/TBymargin 20 def
/Boxing false def
/textshow { ExtendTextBox Gshow } def
%
end
%%EndProlog
%%Page: 1 1
gnudict begin
gsave
doclip
0 0 translate
0.050 0.050 scale
0 setgray
newpath
BackgroundColor 0 lt 3 1 roll 0 lt exch 0 lt or or not {BackgroundColor C 1.000 0 0 8640.00 6048.00 BoxColFill} if
1.000 UL
LTb
LCb setrgbcolor
780 400 M
63 0 V
7435 0 R
-63 0 V
stroke
LTb
LCb setrgbcolor
780 1121 M
63 0 V
7435 0 R
-63 0 V
stroke
LTb
LCb setrgbcolor
780 1842 M
63 0 V
7435 0 R
-63 0 V
stroke
LTb
LCb setrgbcolor
780 2563 M
63 0 V
7435 0 R
-63 0 V
stroke
LTb
LCb setrgbcolor
780 3284 M
63 0 V
7435 0 R
-63 0 V
stroke
LTb
LCb setrgbcolor
780 4005 M
63 0 V
7435 0 R
-63 0 V
stroke
LTb
LCb setrgbcolor
780 4726 M
63 0 V
7435 0 R
-63 0 V
stroke
LTb
LCb setrgbcolor
780 5447 M
63 0 V
7435 0 R
-63 0 V
stroke
LTb
LCb setrgbcolor
780 400 M
0 63 V
0 4984 R
0 -63 V
stroke
LTb
LCb setrgbcolor
1613 400 M
0 63 V
0 4984 R
0 -63 V
stroke
LTb
LCb setrgbcolor
2446 400 M
0 63 V
0 4984 R
0 -63 V
stroke
LTb
LCb setrgbcolor
3279 400 M
0 63 V
0 4984 R
0 -63 V
stroke
LTb
LCb setrgbcolor
4112 400 M
0 63 V
0 4984 R
0 -63 V
stroke
LTb
LCb setrgbcolor
4946 400 M
0 63 V
0 4984 R
0 -63 V
stroke
LTb
LCb setrgbcolor
5779 400 M
0 63 V
0 4984 R
0 -63 V
stroke
LTb
LCb setrgbcolor
6612 400 M
0 63 V
0 4984 R
0 -63 V
stroke
LTb
LCb setrgbcolor
7445 400 M
0 63 V
0 4984 R
0 -63 V
stroke
LTb
LCb setrgbcolor
8278 400 M
0 63 V
0 4984 R
0 -63 V
stroke
LTb
LCb setrgbcolor
1.000 UL
LTb
LCb setrgbcolor
780 5447 N
780 400 L
7498 0 V
0 5047 V
-7498 0 V
Z stroke
1.000 UP
1.000 UL
LTb
LCb setrgbcolor
LCb setrgbcolor
1.000 UL
LTb
0.58 0.00 0.83 C LCb setrgbcolor
1.000 UL
LTb
0.58 0.00 0.83 C 7495 5284 M
543 0 V
780 5322 M
83 -165 V
84 -175 V
83 -184 V
83 -191 V
84 -199 V
83 -203 V
83 -207 V
83 -209 V
84 -210 V
83 -209 V
83 -206 V
84 -203 V
83 -198 V
83 -191 V
84 -183 V
83 -175 V
83 -165 V
84 -153 V
83 -142 V
83 -128 V
84 -116 V
83 -101 V
83 -87 V
83 -72 V
84 -57 V
83 -42 V
83 -27 V
84 -12 V
83 2 V
83 17 V
84 31 V
83 44 V
83 56 V
84 67 V
83 78 V
83 88 V
84 97 V
83 104 V
83 111 V
83 116 V
84 120 V
83 122 V
83 125 V
84 124 V
83 124 V
83 122 V
84 119 V
83 115 V
83 109 V
84 103 V
83 96 V
83 88 V
83 80 V
84 70 V
83 61 V
83 50 V
84 40 V
83 29 V
83 18 V
84 7 V
83 -4 V
83 -14 V
84 -25 V
83 -35 V
83 -44 V
84 -53 V
83 -61 V
83 -69 V
83 -76 V
84 -82 V
83 -87 V
83 -91 V
84 -95 V
83 -97 V
83 -98 V
84 -98 V
83 -98 V
83 -96 V
84 -94 V
83 -91 V
83 -86 V
84 -81 V
83 -75 V
83 -69 V
83 -62 V
84 -54 V
83 -47 V
83 -37 V
84 -29 V
83 -21 V
stroke
LTb
0.00 0.62 0.45 C LCb setrgbcolor
1.000 UL
LTb
0.00 0.62 0.45 C 7495 5084 M
543 0 V
780 4149 M
83 112 V
84 98 V
83 86 V
83 72 V
84 57 V
83 43 V
83 29 V
83 13 V
84 -1 V
83 -16 V
83 -30 V
84 -45 V
83 -58 V
83 -71 V
84 -83 V
83 -95 V
83 -105 V
84 -115 V
83 -124 V
83 -131 V
84 -137 V
83 -143 V
83 -147 V
83 -149 V
84 -151 V
83 -151 V
83 -150 V
84 -148 V
83 -144 V
83 -140 V
84 -134 V
83 -128 V
83 -120 V
84 -111 V
83 -102 V
83 -92 V
84 -81 V
83 -70 V
83 -58 V
83 -47 V
84 -34 V
83 -22 V
83 -10 V
84 2 V
83 14 V
83 26 V
84 36 V
83 48 V
83 57 V
84 67 V
83 75 V
83 82 V
83 90 V
84 95 V
83 100 V
83 104 V
84 107 V
83 109 V
83 109 V
84 109 V
83 108 V
83 105 V
84 102 V
83 97 V
83 93 V
84 86 V
83 80 V
83 72 V
83 64 V
84 56 V
83 47 V
83 37 V
84 28 V
83 18 V
83 9 V
84 -2 V
83 -11 V
83 -20 V
84 -29 V
83 -39 V
83 -46 V
84 -54 V
83 -62 V
83 -67 V
83 -74 V
84 -78 V
83 -83 V
83 -85 V
84 -89 V
83 -89 V
stroke
LTb
0.34 0.71 0.91 C LCb setrgbcolor
1.000 UL
LTb
0.34 0.71 0.91 C 7495 4884 M
543 0 V
780 2977 M
83 78 V
84 82 V
83 86 V
83 88 V
84 89 V
83 89 V
83 90 V
83 88 V
84 85 V
83 83 V
83 79 V
84 73 V
83 68 V
83 62 V
84 54 V
83 47 V
83 38 V
84 29 V
83 20 V
83 10 V
84 1 V
83 -10 V
83 -20 V
83 -30 V
84 -40 V
83 -49 V
83 -60 V
84 -68 V
83 -78 V
83 -85 V
84 -93 V
83 -101 V
83 -106 V
84 -111 V
83 -117 V
83 -120 V
84 -122 V
83 -125 V
83 -125 V
83 -125 V
84 -124 V
83 -122 V
83 -119 V
84 -116 V
83 -110 V
83 -105 V
84 -98 V
83 -91 V
83 -84 V
84 -75 V
83 -66 V
83 -56 V
83 -47 V
84 -37 V
83 -26 V
83 -16 V
84 -6 V
83 4 V
83 15 V
84 24 V
83 34 V
83 44 V
84 52 V
83 59 V
83 68 V
84 74 V
83 80 V
83 86 V
83 89 V
84 94 V
83 96 V
83 97 V
84 99 V
83 98 V
83 97 V
84 96 V
83 93 V
83 89 V
84 85 V
83 79 V
83 74 V
84 67 V
83 61 V
83 53 V
83 45 V
84 36 V
83 28 V
83 20 V
84 10 V
83 2 V
1.000 UP
stroke
LTb
0.90 0.62 0.00 C LCb setrgbcolor
1.000 UP
1.000 UL
LTb
0.90 0.62 0.00 C 780 5322 Box
863 5157 Box
947 4982 Box
1030 4798 Box
1113 4607 Box
1197 4408 Box
1280 4205 Box
1363 3998 Box
1446 3789 Box
1530 3579 Box
1613 3370 Box
1696 3164 Box
1780 2961 Box
1863 2763 Box
1946 2572 Box
2030 2389 Box
2113 2214 Box
2196 2049 Box
2280 1896 Box
2363 1754 Box
2446 1626 Box
2530 1510 Box
2613 1409 Box
2696 1322 Box
2779 1250 Box
2863 1193 Box
2946 1151 Box
3029 1124 Box
3113 1112 Box
3196 1114 Box
3279 1131 Box
3363 1162 Box
3446 1206 Box
3529 1262 Box
3613 1329 Box
3696 1407 Box
3779 1495 Box
3863 1592 Box
3946 1696 Box
4029 1807 Box
4112 1923 Box
4196 2043 Box
4279 2165 Box
4362 2290 Box
4446 2414 Box
4529 2538 Box
4612 2660 Box
4696 2779 Box
4779 2894 Box
4862 3003 Box
4946 3106 Box
5029 3202 Box
5112 3290 Box
5195 3370 Box
5279 3440 Box
5362 3501 Box
5445 3551 Box
5529 3591 Box
5612 3620 Box
5695 3638 Box
5779 3645 Box
5862 3641 Box
5945 3627 Box
6029 3602 Box
6112 3567 Box
6195 3523 Box
6279 3470 Box
6362 3409 Box
6445 3340 Box
6528 3264 Box
6612 3182 Box
6695 3095 Box
6778 3004 Box
6862 2909 Box
6945 2812 Box
7028 2714 Box
7112 2616 Box
7195 2518 Box
7278 2422 Box
7362 2328 Box
7445 2237 Box
7528 2151 Box
7612 2070 Box
7695 1995 Box
7778 1926 Box
7861 1864 Box
7945 1810 Box
8028 1763 Box
8111 1726 Box
8195 1697 Box
8278 1676 Box
7766 4684 Box
1.000 UP
1.000 UL
LTb
0.94 0.89 0.26 C LCb setrgbcolor
1.000 UP
1.000 UL
LTb
0.94 0.89 0.26 C 780 4149 BoxF
863 4261 BoxF
947 4359 BoxF
1030 4445 BoxF
1113 4517 BoxF
1197 4574 BoxF
1280 4617 BoxF
1363 4646 BoxF
1446 4659 BoxF
1530 4658 BoxF
1613 4642 BoxF
1696 4612 BoxF
1780 4567 BoxF
1863 4509 BoxF
1946 4438 BoxF
2030 4355 BoxF
2113 4260 BoxF
2196 4155 BoxF
2280 4040 BoxF
2363 3916 BoxF
2446 3785 BoxF
2530 3648 BoxF
2613 3505 BoxF
2696 3358 BoxF
2779 3209 BoxF
2863 3058 BoxF
2946 2907 BoxF
3029 2757 BoxF
3113 2609 BoxF
3196 2465 BoxF
3279 2325 BoxF
3363 2191 BoxF
3446 2063 BoxF
3529 1943 BoxF
3613 1832 BoxF
3696 1730 BoxF
3779 1638 BoxF
3863 1557 BoxF
3946 1487 BoxF
4029 1429 BoxF
4112 1382 BoxF
4196 1348 BoxF
4279 1326 BoxF
4362 1316 BoxF
4446 1318 BoxF
4529 1332 BoxF
4612 1358 BoxF
4696 1394 BoxF
4779 1442 BoxF
4862 1499 BoxF
4946 1566 BoxF
5029 1641 BoxF
5112 1723 BoxF
5195 1813 BoxF
5279 1908 BoxF
5362 2008 BoxF
5445 2112 BoxF
5529 2219 BoxF
5612 2328 BoxF
5695 2437 BoxF
5779 2546 BoxF
5862 2654 BoxF
5945 2759 BoxF
6029 2861 BoxF
6112 2958 BoxF
6195 3051 BoxF
6279 3137 BoxF
6362 3217 BoxF
6445 3289 BoxF
6528 3353 BoxF
6612 3409 BoxF
6695 3456 BoxF
6778 3493 BoxF
6862 3521 BoxF
6945 3539 BoxF
7028 3548 BoxF
7112 3546 BoxF
7195 3535 BoxF
7278 3515 BoxF
7362 3486 BoxF
7445 3447 BoxF
7528 3401 BoxF
7612 3347 BoxF
7695 3285 BoxF
7778 3218 BoxF
7861 3144 BoxF
7945 3066 BoxF
8028 2983 BoxF
8111 2898 BoxF
8195 2809 BoxF
8278 2720 BoxF
7766 4484 BoxF
1.000 UP
1.000 UL
LTb
0.00 0.45 0.70 C LCb setrgbcolor
1.000 UP
1.000 UL
LTb
0.00 0.45 0.70 C 780 2977 Circle
863 3055 Circle
947 3137 Circle
1030 3223 Circle
1113 3311 Circle
1197 3400 Circle
1280 3489 Circle
1363 3579 Circle
1446 3667 Circle
1530 3752 Circle
1613 3835 Circle
1696 3914 Circle
1780 3987 Circle
1863 4055 Circle
1946 4117 Circle
2030 4171 Circle
2113 4218 Circle
2196 4256 Circle
2280 4285 Circle
2363 4305 Circle
2446 4315 Circle
2530 4316 Circle
2613 4306 Circle
2696 4286 Circle
2779 4256 Circle
2863 4216 Circle
2946 4167 Circle
3029 4107 Circle
3113 4039 Circle
3196 3961 Circle
3279 3876 Circle
3363 3783 Circle
3446 3682 Circle
3529 3576 Circle
3613 3465 Circle
3696 3348 Circle
3779 3228 Circle
3863 3106 Circle
3946 2981 Circle
4029 2856 Circle
4112 2731 Circle
4196 2607 Circle
4279 2485 Circle
4362 2366 Circle
4446 2250 Circle
4529 2140 Circle
4612 2035 Circle
4696 1937 Circle
4779 1846 Circle
4862 1762 Circle
4946 1687 Circle
5029 1621 Circle
5112 1565 Circle
5195 1518 Circle
5279 1481 Circle
5362 1455 Circle
5445 1439 Circle
5529 1433 Circle
5612 1437 Circle
5695 1452 Circle
5779 1476 Circle
5862 1510 Circle
5945 1554 Circle
6029 1606 Circle
6112 1665 Circle
6195 1733 Circle
6279 1807 Circle
6362 1887 Circle
6445 1973 Circle
6528 2062 Circle
6612 2156 Circle
6695 2252 Circle
6778 2349 Circle
6862 2448 Circle
6945 2546 Circle
7028 2643 Circle
7112 2739 Circle
7195 2832 Circle
7278 2921 Circle
7362 3006 Circle
7445 3085 Circle
7528 3159 Circle
7612 3226 Circle
7695 3287 Circle
7778 3340 Circle
7861 3385 Circle
7945 3421 Circle
8028 3449 Circle
8111 3469 Circle
8195 3479 Circle
8278 3481 Circle
7766 4284 Circle
2.000 UL
LTb
LCb setrgbcolor
1.000 UL
LTb
LCb setrgbcolor
780 5447 N
780 400 L
7498 0 V
0 5047 V
-7498 0 V
Z stroke
1.000 UP
1.000 UL
LTb
LCb setrgbcolor
stroke
grestore
end
showpage
  }}%
  \put(7375,4284){\makebox(0,0)[r]{\strut{}$J_2$ (from {\tt<math.h>})}}%
  \put(7375,4484){\makebox(0,0)[r]{\strut{}$J_1$ (from {\tt<math.h>})}}%
  \put(7375,4684){\makebox(0,0)[r]{\strut{}$J_0$ (from {\tt<math.h>})}}%
  \put(7375,4884){\makebox(0,0)[r]{\strut{}$J_2$}}%
  \put(7375,5084){\makebox(0,0)[r]{\strut{}$J_1$}}%
  \put(7375,5284){\makebox(0,0)[r]{\strut{}$J_0$}}%
  \put(4529,5747){\makebox(0,0){\strut{}The Bessel function}}%
  \put(8278,200){\makebox(0,0){\strut{}$10$}}%
  \put(7445,200){\makebox(0,0){\strut{}$9$}}%
  \put(6612,200){\makebox(0,0){\strut{}$8$}}%
  \put(5779,200){\makebox(0,0){\strut{}$7$}}%
  \put(4946,200){\makebox(0,0){\strut{}$6$}}%
  \put(4112,200){\makebox(0,0){\strut{}$5$}}%
  \put(3279,200){\makebox(0,0){\strut{}$4$}}%
  \put(2446,200){\makebox(0,0){\strut{}$3$}}%
  \put(1613,200){\makebox(0,0){\strut{}$2$}}%
  \put(780,200){\makebox(0,0){\strut{}$1$}}%
  \put(660,5447){\makebox(0,0)[r]{\strut{}$0.8$}}%
  \put(660,4726){\makebox(0,0)[r]{\strut{}$0.6$}}%
  \put(660,4005){\makebox(0,0)[r]{\strut{}$0.4$}}%
  \put(660,3284){\makebox(0,0)[r]{\strut{}$0.2$}}%
  \put(660,2563){\makebox(0,0)[r]{\strut{}$0$}}%
  \put(660,1842){\makebox(0,0)[r]{\strut{}$-0.2$}}%
  \put(660,1121){\makebox(0,0)[r]{\strut{}$-0.4$}}%
  \put(660,400){\makebox(0,0)[r]{\strut{}$-0.6$}}%
\end{picture}%
\endgroup
\endinput

		\caption{Comparison of the calculated Bessel functions and the functions from {\tt<math.h>}.}
		\label{fig:plot}
	\end{figure*}
\end{document}
